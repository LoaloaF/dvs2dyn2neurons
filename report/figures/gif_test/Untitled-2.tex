\documentclass[8pt]{beamer}
%gets rid of bottom navigation bars
\setbeamertemplate{footline}[frame number]{}

%gets rid of bottom navigation symbols
\setbeamertemplate{navigation symbols}{}

%gets rid of footer
%will override 'frame number' instruction above
%comment out to revert to previous/default definitions
\setbeamertemplate{footline}{}

\usefonttheme{serif}
\usefonttheme{professionalfonts}

\usepackage{animate}
\usepackage[document]{ragged2e}

% \usepackage{tikz}
% \usetikzlibrary{lindenmayersystems}
% \pgfdeclarelindenmayersystem{A}{%
%   \symbol{F}{\pgflsystemstep=0.6\pgflsystemstep\pgflsystemdrawforward}
%   \rule{A->F[+A][-A]}
% }




% \documentclass[10pt]{beamer}
\usepackage{hyperref}
% \usepackage[margin=1in]{geometry}
\usepackage{amsmath, amssymb}
\usepackage{graphicx}
\usepackage{blindtext}


\usepackage{wrapfig}
\graphicspath{{./figures/}}
\usepackage[T1]{fontenc}

\usepackage{listings}
\usepackage{xcolor}

\definecolor{codegray}{rgb}{0.5,0.5,0.5}
\definecolor{backcolour}{RGB}{240,240,240}

\definecolor{myyellow}{RGB}{255, 226, 36}
\definecolor{myorange}{RGB}{255, 143, 46}
\definecolor{myred}{RGB}{204, 72, 72}

\lstdefinestyle{mystyle}{
	backgroundcolor=\color{backcolour},
    commentstyle=\color{gray},
    keywordstyle=\color{myorange},
    numberstyle=\tiny\color{codegray},
    stringstyle=\color{myred},
    basicstyle=\ttfamily\footnotesize,
    breakatwhitespace=false,         
    breaklines=true,                 
    captionpos=b,                    
    keepspaces=true,                 
    numbers=left,                    
    numbersep=5pt,                  
    showspaces=false,                
    showstringspaces=false,
    showtabs=false,                  
    tabsize=2
}
\lstset{style=mystyle}

% \usepackage{lipsum}
% \usepackage{mwe}
%Document Environment

% there is switches, like \large - can also be encapsulated
% and there are functions like texttt{"some text"}

% \noindent siwtch

% \begin{document}


% \usepackage{animate}
\fontfamily{cmr}\selectfont

\begin{document}
\justify
\linespread{0.8}

\thispagestyle{empty}
\begin{center}
    \vspace{1cm}
    \tiny.\\
		\LARGE \textbf{Selective visual attention for \textit{in vitro} neural stimulation} \\
			
		\vspace{1.8cm}
		\large
		INI-508: Neuromorphic Intelligence \\
		\vspace{0.1cm}
		14.06.2021
			
		\vspace{0.7cm}
		Class Project Report \\
		\textbf{Simon Steffens}
    \pagebreak
	\end{center}

	% \thispagestyle{empty}
	% \pagebreak
	% \tableofcontents
	% \pagebreak

  \tiny .\\
	\large{1. Introduction \& Motivation} \\~\\
	
  {\fontfamily{cmr}\selectfont
  \small{ \textbf{ 1.1 Biohybrid vision implant} \\
  The overarching project inspiring this work is the endeavour of building a
	brain machine interface that uses ectopic axons as electrodes. This approach
	is motivated by the limitations of current technology to deliver high-density
	stimulation in the brain with spatial resolution beyond large neural
	populations. While recording technologies have made considerable progress over
	the last decades, stimulation methods have not kept up. In medical
	applications, deep brain stimulation of basal ganglia has received a lot of
	attention over the last years, especially due to the remarkable improvements
	for patients suffering from Parkinson disease. Still, these systems suffer
	from a range of shortcomings that limit their utility in other settings: first
	and foremost, the spatial resolution of stimulation is limited to neural
	populations or entire nuclei, second, immunoreaction to the implanted
	electrodes causes complications in the long term, and lastly, these systems
	are limited in their adjustability post surgery, as voltage and pulse width
	are the only tunable parameters. Our biohybrid multielectrode array (bioMEA)
	aims to achieve stimulation at single-neuron resolution while simultaneously
	resolving the latent issue of biocompatibility encountered with implanted
	metal electrodes. Such single-cell resolution interfaces are most likely
	required for delivering high dimensional information, for example visual
	input. Our biohybrid interface will be implanted in the dorsal lateral
	geniculate nucleus (dLGN) restoring the visual input as depicted in Figure
	\ref{fig:overview} below.}}













  
\begin{frame}
  \center
  % \animategraphics[loop,autoplay,width=\linewidth]{30}{fra-}{0}{80}
  \animategraphics[loop,autoplay,width=\linewidth]{30}{../simulated_events/frame-}{0}{20}
\end{frame}


\end{document}